\documentclass[a4paper,10pt]{article}
\usepackage[utf8]{inputenc}
\usepackage[T1]{fontenc}
\usepackage[english,italian]{babel}
\usepackage[hmargin=2cm,vmargin=2cm, bottom=2cm]{geometry}
\usepackage{fancyhdr}%
\usepackage{amsfonts}
\usepackage{graphicx}
\usepackage[ruled]{algorithm2e}
\usepackage{titling}
\usepackage[normalem]{ulem}

\setlength{\droptitle}{-5em}   % This is your set screw
\pagestyle{fancy}% Change page style to fancy
\fancyhf{}% Clear header/footer
\fancyhead[C]{Michele Lazzeri | 822879}
\fancyfoot[C]{\thepage}% \fancyfoot[R]{\thepage}
\renewcommand{\headrulewidth}{0.4pt}% Default \headrulewidth is 0.4pt
\renewcommand{\footrulewidth}{0.4pt}% Default \footrulewidth is 0pt
\setlength{\parindent}{0pt}

%inizio custom commands
\newcommand{\entita}[1]{\textsc{\textbf{#1}}}
\newcommand{\assoc}[1]{\textit{#1}}
\newcommand{\relaz}[1]{\textsc{\textbf{#1}}}
\newcommand{\attr}[1]{\textsf{#1}}
\newcommand{\key}[1]{\uline{#1}}
\newcommand{\fkey}[1]{\textit{#1}}
%fine custom commands

\title{Progetto di Basi di Dati - Manuale Utente}
\author{Michele Lazzeri | 822879}
\date{}

\begin{document}
\maketitle
\section{Installazione}
\begin{itemize}
\item Installare PostgreSQL >= 9.1
\item Installare l'ultima versione stabile di Apache
\item Installare PHP >= 5.4
\item Abilitare i moduli PHP di Apache
\item Abilitare i moduli psql di PHP in php.ini
\item Inserire il contenuto della cartella www in una cartella accessibile da Apache
\item Creare un database nuovo tramite psql
\item Inserire il dump (src/dump.sql) in tale database
\item Modificare il contenuto del file www/lib/conf.php inserendo i dati relativi al database
\item Accedere tramite browser al file www/info.php (se i file sono stati copiati in /var/www basta accedere all'indirizzo http://localhost/info.php)
\item Cercare la voce 'include\_{}path' (nel mio caso /usr/share/php)
\item Inserire in tale cartella il contenuto della cartella lib

\end{itemize}
\end{document}