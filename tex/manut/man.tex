\documentclass[a4paper,10pt]{article}
\usepackage[utf8]{inputenc}
\usepackage[T1]{fontenc}
\usepackage[english,italian]{babel}
\usepackage[hmargin=2cm,vmargin=2cm, bottom=2cm]{geometry}
\usepackage{fancyhdr}%
\usepackage{amsfonts}
\usepackage{graphicx}
\usepackage[ruled]{algorithm2e}
\usepackage{titling}
\usepackage[normalem]{ulem}

\setlength{\droptitle}{-5em}   % This is your set screw
\pagestyle{fancy}% Change page style to fancy
\fancyhf{}% Clear header/footer
\fancyhead[C]{Michele Lazzeri | 822879}
\fancyfoot[C]{\thepage}% \fancyfoot[R]{\thepage}
\renewcommand{\headrulewidth}{0.4pt}% Default \headrulewidth is 0.4pt
\renewcommand{\footrulewidth}{0.4pt}% Default \footrulewidth is 0pt
\setlength{\parindent}{0pt}

%inizio custom commands
\newcommand{\entita}[1]{\textsc{\textbf{#1}}}
\newcommand{\assoc}[1]{\textit{#1}}
\newcommand{\relaz}[1]{\textsc{\textbf{#1}}}
\newcommand{\attr}[1]{\textsf{#1}}
\newcommand{\key}[1]{\uline{#1}}
\newcommand{\fkey}[1]{\textit{#1}}
%fine custom commands

\title{Progetto di Basi di Dati - Manuale Utente}
\author{Michele Lazzeri | 822879}
\date{}

\begin{document}
\maketitle
\section{Installazione}
\begin{itemize}
\item Installare PostgreSQL >= 9.1
\item Installare l'ultima versione stabile di Apache
\item Installare PHP >= 5.4 con i relativi moduli php-psql e php-gd
\item Abilitare i moduli PHP di Apache
\item Abilitare i moduli psql di PHP in php.ini
\item Inserire il contenuto della cartella www in una cartella accessibile da Apache
\item Creare un database nuovo tramite psql
\item Inserire il dump (src/dump.sql) in tale database
\item Modificare il contenuto del file www/lib/conf.php inserendo i dati relativi al database
\item Accedere tramite browser al file www/info.php (se i file sono stati copiati in /var/www basta accedere all'indirizzo http://localhost/info.php)
\item Cercare la voce 'include\_{}path' (nel mio caso /usr/share/php)
\item Inserire in tale cartella il contenuto della cartella lib (in modo tale che in /usr/share/php vi sia una cartella chiamata jpgraph)
\item Modificare la password in src/check.sh e rendere tale script eseguibile
\item Creare un nuovo cronjob per tutti i giorni alle 0:01 in cui viene avviata la funzione fix\_{}cron . Per farlo:
\begin{itemize}
\item Da terminale \begin{verbatim}
crontab -e
\end{verbatim}
\item In fondo al file aggiungere:
\begin{verbatim}
1 0 * * * src/check.sh >/dev/null 2>&1
\end{verbatim}
Ovviamente al posto di 'src/check.sh' va inserito il percorso corretto di check.sh
\end{itemize}

\end{itemize}

\section{Note}
\begin{itemize}
\item Per facilitare il testing è stata inserita un form in alto a sinistra nelle pagine web che permette di impostare una data diversa dall'attuale. In questo modo è possibile modificare la data senza dovere cambiare quella di sistema. Tale form ha un solo difetto: essendoci dei controlli ad ogni inserimento di spese per non andare in negativo sul conto, nel momento in cui si inserisce una spesa s1 in un tempo x, quindi si torna ad una data precedente tramite questo form, si inserisce un'altra spesa s2 e si ritorna alla data x, è possibile che quest'ultima operazione non vada a buon fine, non permettendo l'inserimento della spesa s1 in quanto il conto andrebbe in negativo. Per ovviare a tale problema, basta evitare di impostare una data anteriore. Tale comportamente non è comunque vietato dall'applicazione, essendo \emph{esclusivamente} un form di testing. Nell'applicazione reale tale form non dovrebbe esistere, il rinnovo dei conti di credito e dei bilanci avviene tramite cronjob.
\item Il database è popolato da 50 utenti con username 1, 2, 3,... e password 'password'
\item Quasi tutti gli utenti hanno conti di deposito, entrate e spese
\item Solo gli utenti con username 1 e 2 hanno dei conti di credito e bilanci


\end{itemize}
\end{document}